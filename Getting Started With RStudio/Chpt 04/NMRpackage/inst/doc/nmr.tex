\documentclass[12pt]{article}       %% A declaration of type
\usepackage{geometry}               %% A LaTeX package
%
%\VignetteIndexEntry{Using the NMRpackage} %% Meta data lines
%\VignettePackage{NMRpackage}
%\VignetteDepends{zoo}
%
\title{NMRpackage}                  %% A LaTeX macro call
\author{John Verzani}
%
\newcommand{\code}[1]{\texttt{#1}}
\newcommand{\pkg}[1]{\texttt{#1}}
\newcommand{\class}[1]{\texttt{#1}}
\usepackage{Sweave}
\begin{document}                    %% Latex is between begin/end coument
\input{nmr-concordance}
\maketitle                          %% Call a macro to make title
%


This is an example of how one writes a \textit{vignette} for an R
package. Vignettes are typically Sweave documents that get processed
and checked during R's ``build'' and ``check'' invocations.

Sweave can be used to ``weave'' in R code For example, these commands
will be executed and any output (none in this case) would be interspersed:
\begin{Schunk}
\begin{Sinput}
> require(NMRpackage)
> f <- system.file("sampledata","degas.txt", package="NMRpackage")
> a <- readNMRData(f)